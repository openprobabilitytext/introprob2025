% Options for packages loaded elsewhere
\PassOptionsToPackage{unicode}{hyperref}
\PassOptionsToPackage{hyphens}{url}
\PassOptionsToPackage{dvipsnames,svgnames,x11names}{xcolor}
%
\documentclass[
  12pt,
  krantz]{krantzNoCorner}
\usepackage{amsmath,amssymb}
\usepackage{iftex}
\ifPDFTeX
  \usepackage[T1]{fontenc}
  \usepackage[utf8]{inputenc}
  \usepackage{textcomp} % provide euro and other symbols
\else % if luatex or xetex
  \usepackage{unicode-math} % this also loads fontspec
  \defaultfontfeatures{Scale=MatchLowercase}
  \defaultfontfeatures[\rmfamily]{Ligatures=TeX,Scale=1}
\fi
\usepackage{lmodern}
\ifPDFTeX\else
  % xetex/luatex font selection
\fi
% Use upquote if available, for straight quotes in verbatim environments
\IfFileExists{upquote.sty}{\usepackage{upquote}}{}
\IfFileExists{microtype.sty}{% use microtype if available
  \usepackage[]{microtype}
  \UseMicrotypeSet[protrusion]{basicmath} % disable protrusion for tt fonts
}{}
\makeatletter
\@ifundefined{KOMAClassName}{% if non-KOMA class
  \IfFileExists{parskip.sty}{%
    \usepackage{parskip}
  }{% else
    \setlength{\parindent}{0pt}
    \setlength{\parskip}{6pt plus 2pt minus 1pt}}
}{% if KOMA class
  \KOMAoptions{parskip=half}}
\makeatother
\usepackage{xcolor}
\usepackage{longtable,booktabs,array}
\usepackage{calc} % for calculating minipage widths
% Correct order of tables after \paragraph or \subparagraph
\usepackage{etoolbox}
\makeatletter
\patchcmd\longtable{\par}{\if@noskipsec\mbox{}\fi\par}{}{}
\makeatother
% Allow footnotes in longtable head/foot
\IfFileExists{footnotehyper.sty}{\usepackage{footnotehyper}}{\usepackage{footnote}}
\makesavenoteenv{longtable}
\setlength{\emergencystretch}{3em} % prevent overfull lines
\providecommand{\tightlist}{%
  \setlength{\itemsep}{0pt}\setlength{\parskip}{0pt}}
\setcounter{secnumdepth}{5}

\usepackage{booktabs}
\usepackage{empheq}
\setcounter{secnumdepth}{2}

\usepackage{amsfonts}
\usepackage{amsmath}
\usepackage{amssymb}
\usepackage{array}
\usepackage{caption}
\captionsetup[figureNoLabel]{labelformat=empty, justification=raggedright}

\usepackage{color}
\usepackage{colortbl}
\usepackage{xcolor}
\usepackage{xr}


\usepackage{fancyvrb}
\usepackage{framed}
\setlength{\fboxsep}{.8em}

\usepackage{graphicx}

\usepackage{multirow}
\usepackage{tabularx}

\usepackage{setspace}
\usepackage{scalefnt}

\usepackage{verbatim}
\usepackage{pdfpages}
\usepackage{mdframed}

\definecolor{newgray}{rgb}{0.9, 0.9, 0.9}

% Custom environment for background box
\newmdenv[
  backgroundcolor=newgray,
  skipabove=10pt,
  skipbelow=10pt,
  leftmargin=0,
  rightmargin=0,
  innerleftmargin=10pt,
  innerrightmargin=10pt,
  innertopmargin=10pt,
  innerbottommargin=10pt
]{graybox}

\newenvironment{blackbox}{
  \definecolor{shadecolor}{rgb}{0, 0, 0}  % black
  \color{white}
  \begin{shaded}}
 {\end{shaded}}
 
% \newenvironment{greybox}{
%   \definecolor{shadecolor}{rgb}{0.9, 0.9, 0.9}  % light grey??
%   \color{white}
%   \begin{shaded}}
%  {\end{shaded}}
 
% \usepackage[left=1in,top=1in,right=1in,bottom=1in]{geometry}



%\usepackage{wrapfig}



\setlength{\fboxsep}{.8em}

\renewcommand{\hrulefill}{%
  \leavevmode\leaders\hrule height 2pt\hfill\kern0pt }
  
\renewcommand{\dotfill}{%
  \leavevmode\cleaders\hbox to 0.60em{\hss .\hss }\hfill\kern0pt }
  


\usepackage{makeidx}
\makeindex

\ifLuaTeX
  \usepackage{selnolig}  % disable illegal ligatures
\fi
\usepackage[]{natbib}
\bibliographystyle{plainnat}
\IfFileExists{bookmark.sty}{\usepackage{bookmark}}{\usepackage{hyperref}}
\IfFileExists{xurl.sty}{\usepackage{xurl}}{} % add URL line breaks if available
\urlstyle{same}
\hypersetup{
  pdftitle={Introduction to Probability},
  pdfauthor={Madhuka and Rajinda},
  colorlinks=true,
  linkcolor={Maroon},
  filecolor={Maroon},
  citecolor={Blue},
  urlcolor={Blue},
  pdfcreator={LaTeX via pandoc}}

\title{Introduction to Probability}
\author{Madhuka and Rajinda}
\date{2025-04-07}

\begin{document}
\maketitle

{
\hypersetup{linkcolor=}
\setcounter{tocdepth}{2}
\tableofcontents
}
\hypertarget{preface}{%
\chapter*{Preface}\label{preface}}


Placeholder

\hypertarget{fundamentals-of-probability-and-its-axioms}{%
\chapter{Fundamentals of Probability and its Axioms}\label{fundamentals-of-probability-and-its-axioms}}

\hypertarget{combinations-and-permutations}{%
\section{Combinations and Permutations}\label{combinations-and-permutations}}

Counting plays a very important role in probability. In probability, we
often deal with sets, and counting methods such as combinations and
permutations help us find the number of elements in a set. Specifically,
in probability, we deal with a set called the \textbf{sample space} which is
the set of all possible outcomes of some random experiment and
\textbf{events}, which are subsets of the sample space. Typically, the sample
space is denoted by \(S\) or \(\Omega.\) Consider the following example:

\textbf{Example 1.1.} Suppose a fair coin is flipped twice.What is the
probability of flipping at least one head?

\textbf{Solution:}

The sample space of this experiment can expressed as
\(S=\{HH,HT,TH,TT\}\). Now, consider the event:
\[A=\{\text{At least one outcome is a head}\}\] Notice that \(A\subset S\)
(\(A\) is a subset of \(S\)). Since \(A=\{HH,HT,TH\},\) we can find the
probability as follows:
\[P(A)= \frac{\#\text{elements in A}}{\#\text{elements in S}}=\frac{3}{4}.\]

The above method of computing the probability of an event, while very
useful, is not always suitable for the following reasons:

\begin{enumerate}
\def\labelenumi{\arabic{enumi}.}
\tightlist
\item
  Each outcome has to be equally likely.
\item
  The sets must be countable.
\end{enumerate}

From the previous example, it is clear that counting the number of
elements of sets is very important when dealing with discrete sets. We
will do this more efficiently by utilizing tools from combinations and
permutations.

\hypertarget{permutations}{%
\subsection{Permutations}\label{permutations}}

We will begin by stating the following crucial theorem.

\textbf{Theorem 1.1: The Fundamental Principle of Counting}

Consider \(k\) experiments. Let \(n_i\) denote the number of possible
outcomes of the \(i\)th experiment where \(i=1,2,\dots,k.\) Then the total
of number of possible outcomes is
\[n_1\times n_2\times\cdots\times n_k.\] \textbf{Example 1.2.} A 6-sided die
is rolled twice. How many possible outcomes are there?

\textbf{Solution:}

First roll has 6 outcomes and the second roll has 6 outcomes. Therefore,
there are \(6\times 6 = 36\) possible outcomes when a dice rolled twice.

In general, if a 6-die is rolled \(n\) times, then there are \(6^n\)
possible outcomes.

\textbf{Example 1.3.} Suppose a password must contain exactly 4 symbols,
where each symbol is either a letter (a-z) or a number (0-9). How many
passwords can be created,

\begin{enumerate}
\def\labelenumi{\arabic{enumi}.}
\tightlist
\item
  if the password is not case-sensitive?
\item
  if the first two symbols must be numbers?
\item
  if the first two symbols must be numbers and the rest should be
  letters?
\end{enumerate}

\textbf{Solution:}

\begin{enumerate}
\def\labelenumi{\arabic{enumi}.}
\tightlist
\item
  If the password is not case-sensitive, then there are
  \(36\times36\times36\times36\) different passwords.
\item
  If the first two symbols must be numbers, then there are
  \(10\times10\times36\times36\) different passwords.
\item
  If the first two symbols must be numbers and the rest should be
  letters, then there are \(10\times10\times26\times26\) different
  passwords.
\end{enumerate}

\textbf{Exercise 1.1.} Repeat the above example assuming the password is
case-sensitive.

\textbf{Example 1.4:} Suppose a password must contain exactly 4 symbols,
where each symbol is either a letter (a-z) or a number (0-9). How many
passwords contain at least one letter?

\textbf{Solution:}

Total number of passwords = \(36^4\). Total number of passwords not
containing a letter = \(10^4\). Therefore, the number of passwords
containing at least one letter is \[36^4-10^4.\]

Imagine three objects labeled \(A,B,\) and \(C.\) The number of ways we can
arrange these objects is called a \emph{permutation.} These objects can be
arranged in six different ways: \[ABC, ACB, BAC, BCA, CAB, CBA.\] The
formula to find the number of permutations of \(n\) \emph{distinct} objects is
\[n!=n\times (n-1)\times\cdots\times 2\times 1.\] For three distinct
objects, the number of permutations is \(3!=6.\)

\textbf{Example 1.5.} In a class, there are 4 sophomores, 7 juniors, and 3
seniors.

\begin{enumerate}
\def\labelenumi{\arabic{enumi}.}
\tightlist
\item
  How many ways are these students be arranged in a row?
\item
  How many ways can these students be arranged if all the seniors
  should remain together?
\end{enumerate}

\textbf{Solution:}

\begin{enumerate}
\def\labelenumi{\arabic{enumi}.}
\tightlist
\item
  \(12!\)
\item
  Since the seniors must stand together, we think of the seniors as
  one ``block''. Therefore, together with this block, 4 sophomores,
  and 7 juniors, there are 12 objects to be arranged. We can also
  arrange the three seniors within the block in \(3!\) ways. Therefore,
  the total number of permutations is:
\end{enumerate}

\includegraphics[width=0.8\linewidth,height=0.5\textheight]{class_example_permutations}

\textbf{Exercise 1.2.} How many ways can the students be arranged if students
from the same year should remain together?

\textbf{Exercise 1.3.} How many ways can the students be arranged if two
particular students do not wish to stand together?

\textbf{Exercise 1.4.} How many ways can the students be arranged if at most
two seniors should be together?

Now let's learn how to find the permutations of objects that are
non-distinct (some are identical). Consider the permutations of AAB.
Since we have two identical objects, the number of permutations has to
be less than \(3!\) (since permuting the two ``A''s gives us the same
arrangement).

\textbf{Theorem 1.1: Permutations of non-identical objects}

Consider a set of \(n\) objects of which \(n_1\) objects are alike, \(n_2\) objects are alike, \(\dots\), \(n_k\) objects are alike. Then, the number of permutations of these \(n\) objects is given by \[\frac{n!}{n_1!n_2!\cdots n_k!}\]

\textbf{Example 1.6.} Find the number of ways the following words can be
permuted:

\begin{enumerate}
\def\labelenumi{\arabic{enumi}.}
\tightlist
\item
  MATH
\item
  BEEP
\item
  COCO
\item
  SUCCESS
\end{enumerate}

\textbf{Solution:}

\begin{enumerate}
\def\labelenumi{\arabic{enumi}.}
\tightlist
\item
  4 different letters. Therefore, number of permutations: \[4!.\]
\item
  4 letters word, 2 letters are alike. Therefore, number of
  permutations: \[\frac{4!}{2!}.\]
\item
  4 letters word, 2 letters are alike and another 2 letters are alike.
  Therefore, number of permutations: \[\frac{4!}{2!2!}.\]
\item
  7 letters word, 3 letters are alike and another 2 letters are alike.
  Therefore, number of permutations: \[\frac{7!}{3!2!}.\]
\end{enumerate}

In each of the above questions, we make an adjustment for over-counting
by diving by the permutations of identical objects.

\hypertarget{combinations}{%
\subsection{Combinations}\label{combinations}}

Combinations are the number of ways a subset can be chosen from a larger
set where the order of the selection does not matter. Consider a set of
\(n\) distinct objects. The number of ways a subset of \(r\leq n\) objects
can be chosen can be computed using the formula
\[{n\choose r} = \frac{n!}{r!(n-r)!}.\] The key point here is that,
unlike in permutations, the order of the \(r\) objects selected is
irrelevant.

\textbf{Example 1.7.} Suppose a shelf contains 10 distinct books out of which
4 are math books and the rest are physics books.

\begin{enumerate}
\def\labelenumi{\arabic{enumi}.}
\item
  How many ways can 3 books be chosen where the order does not matter?
\item
  How many ways can 4 books be chosen so that exactly two are math
  books?
\end{enumerate}

\textbf{Solution:}

\begin{enumerate}
\def\labelenumi{\arabic{enumi}.}
\item
  \(10\choose 3\)
\item
  The number of ways two math books can be chosen = \(4\choose 2\)\textbackslash{} The
  number of ways two other books (non-math) can be chosen =
  \(6\choose 2\)\textbackslash{}
\end{enumerate}

Therefore, the total number of ways exactly two math books can be chosen
is \[{4\choose 2}{6\choose 2}.\]

\textbf{Exercise 1.5.} How many ways can at least one math book be chosen?

\textbf{Exercise 1.6.} How many ways can at most one math book be chosen?

Some questions in this course involves questions based on a deck of
cards. This is the composition of a deck of cards:

\begin{figure}
\includegraphics[width=0.9\linewidth,height=0.4\textheight]{deckofcards} \caption{Illustration of a standard deck of cards. Source: Missouri Western State University (https://intranet.missouriwestern.edu/cas/wp-content/uploads/sites/17/2020/05/Standard-Deck-of-Cards.pdf)}\label{fig:ExCardDeck}
\end{figure}

In the Figure \texttt{\ref{fig:ExCardDeck}}, the columns represent the
different \textbf{values} or \textbf{denominations} and the rows represent the
four \textbf{suits.} The last three values, Jack, Queen, and King are called
\textbf{face cards.}

\textbf{Example 1.8.} Consider a deck of 52 cards.

\begin{enumerate}
\def\labelenumi{\arabic{enumi}.}
\tightlist
\item
  How many ways can 5 cards be chosen?
\item
  How many ways can 5 cards be chosen if exactly 2 cards must be
  Queens?
\item
  How many ways can 5 cards be chosen if exactly 2 Queens and 2 Kings
  must be chosen?
\item
  How many ways can we choose 5 cards such that exactly two have the
  same value?
\end{enumerate}

\textbf{Solution:}

\begin{enumerate}
\def\labelenumi{\arabic{enumi}.}
\item
  \(\binom{52}{5}\)
\item
  A deck of cards has 4 Queens. We first choose two Queens from the
  set of 4 Queens and 3 additional cards are chosen from the remaining
  48 cards (since 52-4 = 48). Therefore, the answer is
  \[\binom{4}{2}\binom{48}{3}.\]
\item
  \(\binom{4}{2}\binom{4}{2}\binom{44}{1}\)
\item
  There are 13 different values in a deck of cards: Ace, 2, 3, \ldots,
  Q, K. Each value appears on 4 cards. To choose two cards from the
  same value, we first pick a value. This can be done in
  \(\binom{13}{1}=13\) ways. Thereafter, we choose 2 cards from the 4
  cards of this chosen value. This can be done in \(\binom{4}{2}\) ways.
  Now we choose 3 additional cards that are different from this value.
  However, we also must make sure that we do not pick more than one
  card from another value. Out of the 12 values left, we first choose
  three values (\(\binom{12}{3}\)ways), and then from each value, we
  choose a card (\(\binom{4}{1}^3=4^3\) ways). Therefore, the final
  answer is \[13\cdot\binom{4}{2}\cdot \binom{12}{3}\cdot 4^3.\]
\end{enumerate}

\textbf{Exercise 1.7.} How many ways can five cards be chosen such that all
the values are unique?

\textbf{Exercise 1.8.} How many ways can five cards be chosen such that two
cards are of one value and three cards of another value? (For example,
QQAAA,22299,etc.).

\hypertarget{worked-examples}{%
\subsection{Worked Examples}\label{worked-examples}}

\begin{enumerate}
\def\labelenumi{\arabic{enumi}.}
\item
  How many different ways can 5 cards be chosen from a deck of 52 if
  exactly two of the cards are of the same value and the other three
  cards are of different values? (Example: AA5JQ, 66KQ2, etc.)
\item
  How many ways can 5 cards be chosen from a standard 52-card deck if
  all 5 cards must have different values? (Ex: A2JQ3, 7J2Q4, etc.)
\item
  A class consists of 40 seniors and 60 juniors. The teacher wants to
  form a committee consisting of 4 seniors and 6 juniors.

  \begin{enumerate}
  \def\labelenumii{\alph{enumii}.}
  \tightlist
  \item
    How many ways can the committee be formed?
  \item
    How many ways can the committee be formed if two specific
    seniors do not wish to serve on the committee together?
  \item
    Greg is one of the seniors in the class. How many different
    committees can be formed that include Greg as a member?
  \end{enumerate}
\item
  Ten identical pieces of candy are to be distributed between four
  students - A,B,C,and D.

  \begin{enumerate}
  \def\labelenumii{\alph{enumii}.}
  \tightlist
  \item
    How many ways can the candy be distributed?
  \item
    How many ways can the candy be distributed if \(A\) must receive
    exactly two pieces?
  \item
    How many ways can the candy be distributed if A must receive at
    least two pieces?
  \end{enumerate}
\item
  All the cards of deck of 52-cards are arranged in a line.

  \begin{enumerate}
  \def\labelenumii{\alph{enumii}.}
  \tightlist
  \item
    How many ways can the cards be arranged such that the colors
    (red and black) alternate?
  \item
    How many ways can the cards be arranged if not all four Queens
    are together?
  \end{enumerate}
\end{enumerate}

\textbf{Solution:}

\begin{enumerate}
\def\labelenumi{\arabic{enumi}.}
\item
  A standard deck of cards has 13 different valueS. We need to first
  choose a value and then pick two cards from that value. Since each
  value has 4 cards, we can do this in \(\binom{13}{1}\binom{4}{2}\)
  ways. Thereafter, the other three cards must be picked from three
  other values. Since we have already picked 2 cards from one of the
  values, there are 12 values left to choose from. We first pick 3
  values and then choose a card from each value. This can be done on
  \(\binom{12}{3}\binom{4}{1}\binom{4}{1}\binom{4}{1}\) ways. Therefore,
  the final answer is
  \[\binom{13}{1}\binom{4}{2}\binom{12}{3}\binom{4}{1}\binom{4}{1}    \binom{4}{1}.\]
\item
  This is a little bit similar to the previous problem, but here we
  have to choose 5 distinct values. We first choose 5 values and then
  from each value, we pick a card. Therefore, the final answer is
  \[\binom{13}{5}\binom{4}{1}\binom{4}{1}\binom{4}{1}\binom{4}{1}\binom{4}     {1}=\binom{13}{5}\cdot 4^5.\]
\item
  a. \(\binom{40}{4}\binom{60}{6}\)

  b. We will find the answer to this using an indirect method. Let's
  first find the number of committees where the two specific students
  \textit{are} together. The set of seniors consists of these 2
  specific students and 38 other seniors. The number of ways we can
  choose 4 seniors where two specific students are always included in
  the committee is \(\binom{2}{2}\binom{38}{2}=\binom{38}{2}.\) Since
  there are no restrictions for the juniors, the total number of
  committees where these two specific students are in the committee is
  \(\binom{38}{2}\binom{60}{6}.\) Since the total number of committees
  is \(\binom{40}{4}\binom{60}{6}\) (this is what we found in part (a)),
  we can find the answer to this question by subtracting the number of
  committees where the two students are together from the total number
  of committees (this gives us the number of committees where only one
  of them are together or none of them are in the committee):
  \[\binom{40}{4}\binom{60}{6}-\binom{38}{2}\binom{60}{6}.\]
\end{enumerate}

c.
\[\binom{1}{1}\binom{39}{3}\binom{60}{6}=\binom{39}{3}\binom{60}{6}.\]
4. a. Let's use \(X\) to represent a piece of candy. We can use three
lines (dividers) to divide the candy between the three students. For
example \[X|XX|XXX|XXXX\] means A gets 1, B gets 2, C gets 3, and D gets
4. Similarly, \[X||XX|XXXXXXX\] means A receives 1, B receives none, C
receives 2, and D receives 7. Therefore, we can find the total number of
distributions by finding the total number of arrangements of the Xs and
the lines. Since there are 10 Xs and three lines, but the 10 Xs are
identical and the 3 lines are identical (total 13 objects), the total
number of ways the candy can be distributed is
\[\frac{13!}{3!10!}=\binom{13}{3}.\]

\begin{enumerate}
\def\labelenumi{\alph{enumi}.}
\setcounter{enumi}{1}
\item
  \[\underbrace{XX}_{A\text{ receives 2 
   }}\underbrace{XXXXXXXX}_{\text{Distribute the rest between B,C,D}}.\]
  This problems simplifies to 8 pieces of candy and three students -
  B,C,D. To distribute between 3 students, we need 2 lines. Therefore,
  the total number of distributions is
  \[\frac{10!}{2!8!}=\binom{10}{2}.\]
\item
  This is similar to the previous part. Initially, we give A two
  pieces of candy. However, when we distribute the other 8 pieces of
  candy, we distribute between all the 4 students. This way, A gets at
  least 2. \[\underbrace{XX}_{A\text{ receives 2 
   }}\underbrace{XXXXXXXX}_{\text{Distribute the rest between A,B,C,D}}.\]
  Therefore, we have 8 Xs and 3 lines (total 11). The answer is
  \[\frac{11!}{3!8!}=\binom{11}{3}.\] 5.a. We can arrange the cards
  starting with a red card (RBRB\ldots) or with a black card (BRBR\ldots).
  Since there are 26 red cards and 26 black cards, the answer is
  \[ 26!26!\cdot2.\] b. The total number of arrangements without any
  restrictions is 4!. The number of ways we arrange the cards keeping
  \textit{all} the queens together is \(4!49!\) (48 other cards + one
  block of 4 queens = 49 objects). Therefore, the number of
  arrangements where at least one queen is separated is \[52!-49!4!.\]
\end{enumerate}

\hypertarget{basic-probability}{%
\chapter{Basic Probability}\label{basic-probability}}

\end{document}
